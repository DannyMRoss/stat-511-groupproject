\documentclass[11pt]{article} 

\usepackage[margin=1in]{geometry}
\usepackage{amssymb}
\usepackage{fancyhdr}

\pagestyle{fancy}
\fancyhf{}
\fancyhead[R]{\textbf{Chengyuan, Daniel, and Junjie}}
\fancyhead[L]{\textbf{STAT 511 - Group Project Proposal}}
\fancyfoot[C]{\thepage}
\setlength{\headheight}{15pt}

\newcommand{\teh}{treatment effect heterogeneity}
\newcommand{\hte}{heterogeneous treatment effects}

\begin{document}

\section{Problem Description and Modeling Objective}
%Provide a clear statement of the problem being investigated in the paper. Clearly state the specific objective of the statistical modeling.

In the paper ``Estimating treatment effect heterogeneity in randomized program evaluation,''\cite{imai_estimating_2013} the authors are concerned with ``treatment effect heterogeneity'' which they define as ``the degree to which different treatments have differential causal effects on each unit.'' 
The authors' objective is to estimate \teh\ in order to (1) select the most effective treatment among a large number of available treatments, (2) design optimal treatments for sub-groups of units, (3) test the existence of \teh, and (4) generalize causal effect estimates from a sample to a target population.

\section{Data Description and Availability of Dataset}
%Provide a detailed overview of the dataset used (from the main application), including information about the variables in the dataset.

The \texttt{R} package \texttt{FindIt} includes the data from two well-known randomized evaluation studies in the social sciences that the authors' apply their model to.\cite{FindIt2012} 
Including the dataset \texttt{GerberGreen}, which is data from the 1998 New Haven Get-Out-the-Vote filed experiment where many different mobilization techniques were randomly administered to voters in the 1998 election.
As well as the dataset \texttt{LaLonde}, which is data from the national supported work (NSW) program that was a job training program intended to increase earnings of workers conducted from 1975 to 1978 over 15 sites in the United States.

\texttt{GerberGreen} includes a binary outcome variable of whether a registered voter voted or not in the 1998 election, four factor variables that combine to form 193 unique treatment combinations, and pre-treatment control covariates including age, party affiliation, 1996 voting indicator. 
\texttt{LaLonde} includes a binary outcome variable of whether earnings in 1978 are larger than in 1975, a binary treatment variable, and pre-treatment control covariates including age, education, race, and earnings.


\section{Model and Methods Description}
%Clearly describe the proposed model in the analysis. Outline the statistical estimation methods.

In order to overcome the methodological challenges of (1) extracting useful information from sparse randomized evaluation study data, (2) identifying sub-groups for whom a treatment is beneficial, and (3) generalizing the results of an experiment to a target population, the authors' formulate the estimation of heterogeneous treatment effects as a variable selection problem.
Specifically, the paper develops a Squared Loss Support Vector Machine (L2-SVM) with separate LASSO constraints over the pre-treatment and causal heterogeneity parameters, such that the causal heterogeneity variables of interest are separated from the rest of the variables. 

The proposed model is grounded within the potential outcomes framework for causal inference.
In this framework, the causal effect of treatment $t$ for unit $i$ is defined as $Y_i(t) - Y_i(0)$, where $Y_i$ is the potential outcome for unit $i$ under treatment or control.
Thus, by leveraging the fact that the L2-SVM is an optimal classifier, the proposed model can estimate \hte\ by predicting the potential outcomes $Y_i(t)$ directly from the fitted model and estimate the conditional treatment effect as the difference between the predicted outcome under treatment status $t$ and under the control condition.

To fit the proposed model the authors' use an estimation algorithm based on a generalized cross-validation (GCV) statistic.
Because of the structure of the proposed model, the SVM becomes a least squares problem on a subset of the data, therefore the L2-SVM is fitted through a series of iterated LASSO fits.
Accordingly, the authors' employ an efficient coordinate descent algorithm for the LASSO fits. 


% Bibliography ----------------------
\bibliographystyle{plain}
\bibliography{../bibliography}

\end{document}  