\documentclass[12pt]{article} 

\usepackage[margin=1in]{geometry}
\usepackage{amssymb}
\usepackage{fancyhdr}

\pagestyle{fancy}
\fancyhf{}
\fancyhead[R]{Chengyuan, Daniel, and Junjie}
\fancyhead[L]{Group Project Proposal}
\fancyfoot[C]{\thepage}
\setlength{\headheight}{15pt}

\newcommand{\teh}{treatment effect heterogeneity}

\begin{document}

\section{Problem Description and Modeling Objective}
%Provide a clear statement of the problem being investigated in the paper. Clearly state the specific objective of the statistical modeling.

In the paper ``Estimating treatment effect heterogeneity in randomized program evaluation,''\cite{imai_estimating_2013} the authors are concerned with ``treatment effect heterogeneity'' which they define as ``the degree to which different treatments have differential causal effects on each unit.'' Estimation of \teh\ is also important when (1) selecting the most effective treatment among a large number of available treatments, (2) designing optimal treatments for sub-groups of units, (3) testing the existence of \teh, and (4) generalizing causal effect estimates from a sample to a target population.


\section{Data Description and Availability of Dataset}
%Provide a detailed overview of the dataset used (from the main application), including information about the variables in the dataset.

The \texttt{R} package \texttt{FindIt} includes the data analyzed in the paper.\cite{FindIt2012}


\section{Model and Methods Description}
%Clearly describe the proposed model in the analysis. Outline the statistical estimation methods.


% Bibliography ----------------------
\bibliographystyle{plain}
\bibliography{../bibliography}

\end{document}  